\documentclass[11pt,a4paper]{jsarticle}
\usepackage[dvips]{graphicx}
\usepackage{fancyhdr}
\usepackage{here}
\setcounter{page}{0}
%
\begin{document}

\section{自己位置推定}
ロボカーの中心が点$(x_k,y_k)$の位置にあるとし、時間$\Delta t$秒の間にロボカーが旋回中心の周りに$\Delta \theta$だけ回転したとする.このときロボカーの中心から車輪までの距離を$d$,左右の車輪とロボカーの中心が動いた距離をそれぞれ$\Delta L_R, \Delta L_L, \Delta L$とすると

\begin{eqnarray}
 \Delta L_R & = & (\rho + d)\Delta \theta \nonumber \\ 
 \Delta L_L & = & (\rho - d)\Delta \theta \nonumber \\
 \Delta L & = &\rho \Delta \theta
\end{eqnarray}

となる.これより

\end{document}