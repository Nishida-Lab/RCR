\documentclass[11pt,a4j]{jarticle}


\begin{document}
\section{モデル化}
プログラム上でセンサから読み取った値を扱えるようにするため,モデル化を行った.
ここでの距離,加速度,電圧の単位はそれぞれcm,$\mathrm{m}/\mathrm{s}^{2}$,Vである.

\subsection{PSDセンサ GP2Y0A21}
前回求めたPSDセンサの性能から距離-電圧特性は以下のようになる.
\begin{equation}
 (距離)=45.514×(電圧)^{-0.822}
\end{equation}

\subsection{近接センサ VL6180X}
前回のPSDセンサと同様な実験で,近接センサの距離と出力値の関係を求めた.
そのときの関係を表した式を以下に示す.
また,出力値とはarduinoで0〜255までの値で出力された値であり,
対象物との距離が近いほど,値が小さくなるようにプログラムで処理している.
\begin{equation}
 (距離)=0.09999×(出力値)+0.4477
\end{equation}

\subsection{加速度センサ KXR-94 }
この加速度センサのモデル化するために必要な仕様を以下に示す.
\begin{itemize}
 \item 感度:1 [V/g]
 \item オフセット:2.5 [V]
\end{itemize}

これらより,加速度と電圧の関係を以下に示す.
\begin{equation}
 (加速度)=9.8×[(電圧)-2.5]
\end{equation}



\section{自己位置推定}
\subsection{加速度センサ}
加速度センサを用いて,移動距離を求める.
そのために加速度センサから,加速度とそのときの時間を得る.
今回は加速度を積分するのに,数値積分のシンプソン公式を用いた.
シンプソン公式による区間$(a,b)$の$f(t)$の積分値は以下のようになる.


\begin{equation}
 \int_a^b f(t) dt = 
\frac{b-a}{6}\left(f(a)+4f\left(\frac{a+b}{2} \right)+f(b) \right)
\end{equation}

これを用いて,$(t_0,t_2)$の区間で積分する(ただし$t_1=\frac{t_0+t_2}{2}$).\\
$a(t)[\mathrm{m}/\mathrm{s}^{2}]$を加速度,$v(t)$[m/s]を速度とすると

\begin{equation}
 v(t)= \frac{t_2-t_0}{6}\left(a(t_0)+4a(t_1)+a(t_2) \right)
\end{equation}

となる.よって移動距離を$x(t)$[m]とすると以下のようになる.
\begin{equation}
 x(t)= \frac{t_2-t_0}{6}\left(v(t_0)+4v(t_1)+v(t_2) \right)
\end{equation}

\subsection{ジャイロセンサ}
ジャイロセンサ(角速度センサ)を用いて,車体の角度を求める.
角速度センサと同様に考える.
角度を$\theta(t)$[deg],角速度を$\omega(t)$[deg/s]とすると

\begin{equation}
 \theta(t)=\frac{t_2-t_0}{6}\left(\omega(t_0)+4\omega(t_1)+\omega(t_2) \right)
\end{equation}
となる.

\end{document}