\documentclass[10pt,a4j]{jarticle}

\input{include/macro.tex}
\input{include/preamble.tex}

\begin{document}
\section{制御系}
\subsection{角度ベースの人工ポテンシャル法}
本章では,Huangらによって提案された
角度ベースの人工ポテンシャル法\cite{Huang}について概説する.
この手法は次の角運動方程式によって運動の制御が行われている.

\begin{equation}
 \ddot{\phi}=-b\dot{\phi}-k_g(\phi-\psi_g)(\mathrm{e}^{-c_1d_g}+c_2)+
 \sum_i k_{o_i}(\phi-\psi_{o_i})\mathrm{e}^{c_3|\phi-\psi_{o_i}|}
 \cdot \mathrm{e}^{-c_4d_{o_i}}
 \label{potential_1}
\end{equation}

ここで$b,k_g,k_{o_i},c_1,c_2,c_3,c_4は定数であり,\phi$は
ロボットの角度,$\phi_g,d_g$はそれぞれロボットから見た目的地の方向と距離,
$\phi_o,d_o$はそれぞれロボットから見た障害物の方向と距離を示す.

(\ref{potential_1})式は3つの項から成り立っている.

第一項は減衰項であり,減衰定数bによってロボットの挙動の安定化を図っている.

第二項はをロボットの速度方向を目的地方向へと向けるトルクを表す.
このトルクは目的地との距離$d_g$が0に近づくにつれて,
大きくなり,ある一定の値に近づく.

第三項はロボットの速度方向をそれぞれの障害物方向から
遠ざけようとするトルクの和を表す.これらのトルクは障害物との距離$d_o$が
0に近づくにつれて,大きくなり,ある一定の値に近づく.
\\
第二項と第三項の和はポテンシャル力として表すことができ,
(\ref{potential_1})式は(\ref{potential_2}),(\ref{potential_3})式
のように表すことができる.

\begin{equation}
 \ddot{\phi}=\frac{\Phi}{\phi}-b\dot{\phi}
 \label{potential_2}
\end{equation}
\begin{equation}
 \Phi=\Phi_g+\sum_i \Phi_{o_i}
 \label{potential_3}
\end{equation}
\\ここで$\Phi$はポテンシャル(位置エネルギー)である.(\ref{potential_3})式の第一項はゴールポテンシャルを表しており,目的地方向を
ロボットが向いている時に最小値をとる.また,(\ref{potential_3})式の第二項は
障害物ポテンシャルを表しており,ロボットの向きが障害物の方向に近づくほど,
また,ロボットと障害物の距離が小さくなるほど大きくなる.

この他,Huangらの手法では,速度方向の制御と同時に以下の式によって
速度の制御も行っている.
\begin{equation}
 v=\mathrm{max}(v_{MAX}\mathrm{e}^{-k_v\Phi_o}-\varepsilon,0) 
\end{equation}

ここで$k_vは定数であり,\varepsilon$は充分小さい定数を表す.

これは障害物ポテンシャルが大きくなるほど,速度が小さくすることを表している.
このため,Huangらの手法では障害物との衝突の危険性が高い領域では
速度を落として衝突回避を行っている.


\begin{thebibliography}{2}
 \bibitem{Huang} H.Huang,R.Fajen,R.Fink,H.Warren
``Visual navigation and obstacle avoidance using a steering potential 
function'',Robotics and Autonomous Systems 54 pp.288-299(2006)
\end{thebibliography}

\end{document}