\documentclass[10pt,a4j]{jarticle}

\begin{document}
\section{フィルタ処理}
\subsection{ローパスフィルタ}
加速度センサやジャイロセンサで読み取った値は,ノイズが入り大きく上下していた.
そこでセンサ値の平滑化を行うため,ローパスフィルタを用いた.ローパスフィルタの式は以下に示す.

\begin{equation}
 y[i]=py[i-1]+(1-p)x[i]
 \label{low_pass}
\end{equation}
ここで,$y[i]$は出力値,$y[i-1]$は前回の出力値,$x[i]$は現在のセンサ値であり,
$p(0<p<1)$はパラメータである.この式の特徴として,パラメータ$p$を大きくすれば滑らかになるが,
位相が遅れることがわかっている.

\subsection{ハイパスフィルタ}
センサ値をローパスフィルタで平滑化を行い,移動距離や回転角度を求めるためにその値の積分を行った.
ここでの積分は数値積分であったため,積分誤差が生じ,時間とともに値がずれていた.
これを解決するためにハイパスフィルタを用いた.ハイパスフィルタの式は以下に示す.

\begin{equation}
 \acute{y}[i]=x[i]-y[i]
 \label{high_pass}
\end{equation}
ここで,$\acute{y}[i]$は出力値,$x[i]$は現在の積分したセンサ値,
$y[i]$は積分したセンサ値をローパスフィルタに通した出力値である.
\end{document}