\documentclass[10pt,a4j]{jarticle}

\input{include/macro.tex}
\input{include/preamble.tex}

\begin{document}
\section{画像処理}
画像処理のフローについて実際の処理画像とともに説明をする.

\begin{enumerate}
 \item 元画像(RGB)
 \item 元画像(RGB)から上半分を削除
 \item HSV変換をしてH(色相)だけ取り出し,グレースケールで表示
 \item 色相を180度回転
 \item 赤の色相を強調するフィルタ処理を行う
 \item 上のフィルタ処理を5回反復
 \item ノイズ除去を行う
 \item エッジを抽出
 \item エッジから輪郭線を取り出す
 \item 前の画像から最大面積の輪郭線を出す
 \item 重心を求める
\end{enumerate}

\begin{figure}[h]
  \begin{flushleft}
    \includegraphics[width=0.9\hsize]{image_flow.eps}
    \caption{画像処理のフロー}
    \label{image_flow}
  \end{flushleft}
\end{figure}

\end{document}