\documentclass[10pt,a4j]{jarticle}

\input{include/macro.tex}
\input{include/preamble.tex}

\begin{document}
\section{消火ポールの構造の変更}
前回の消火ポールを実際に作成してみて以下の欠点が見つかった.
\begin{itemize}
 \item 	赤い布で青ポールを綺麗に隠せない
 \item  布を押し込む力が弱いとポールの中に入らない
 \item  布の中心を押し込む必要がある
 \item  布が長すぎてポールの中に入っても外に飛び出してしまう
\end{itemize}
そこで,当初考えていたポールの周りに短冊状の裏表が赤と青の紙を取り付け,
紙が自重によって裏返ることで消火とする.消火ポールを図\ref{poll}に示す.
また,紙を上で支えるため,押し込む力が弱くても大丈夫であるが,
固定するには十分なマグネットシートとクリップを使用している.
これにより前回の消火ポールの欠点を全て克服出来るポールを作成した.

\begin{figure}[hb]
  \begin{center}
    \includegraphics[width=0.5\hsize]{poll.eps}
    \caption{消火ポール}
    \label{poll}
  \end{center}
\end{figure}
\end{document}